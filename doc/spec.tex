\documentclass[a4paper, 11pt, leqno]{article}
\usepackage{amsmath, amsfonts, amssymb}
\usepackage{graphics}
\usepackage{color}
\usepackage{fancyhdr} 
%\usepackage{sectsty}
\usepackage[T1]{fontenc}
%\usepackage{pstricks}
%\usepackage{mathrsfs}
%\usepackage{pifont}
\usepackage{array}
%\usepackage{titlesec}

%% pour les fontes
\usepackage{ae}
\usepackage{aecompl}

%% hyperref avec bookmarks
%%\usepackage[bookmarks=true,hyperindex=true,
%%pdftitle={On the Entropy of Braids},
%%colorlinks=true,linkcolor=blue,
%%pdfauthor={J.-O. Moussafir},
%%pdfstartview=FitH]{hyperref} 
 
%%\DeclareCaptionLabelSeparator{colon}{ -- }
%%\captionsetup{textfont=it, labelfont=bf}


\begin{document}

Avec un pas de temps de 1 jour, je note $S_t^i$ la production électrique 
d'une unité photo-voltaique sur le site $i$, et $E_t^i$ la production d'une unité éolienne 
sur le site $i$ à la date t. C'est la donné du pb, $S_t^i$ et $E_t^i$ 
dépendent des conditions météo moyennes selon la période de l'année etc. 

Les inconnues du problème sont $p_i$, le nombre d'unités photo-voltaiques sur le site $i$
et $q_i$ le nombre d'éoliennes sur le site $i$. On veut que chaque jour
$$
P_t =\sum_i p_i S_t^i + q_i E_t^i \geq A
$$
et aussi que la variance de $P_t$ soit aussi petite que possible, c'est-à-dire qu'on veut 
résoudre
\begin{align*}
& \min \sum_t \left( P_t - \langle P \rangle \right)^2 \quad \text{avec} \\
& P_t \geq A \text{ à chaque t}.
\end{align*}
où $\langle P \rangle$ désigne la moyenne de $P_t$.

On pourrait aussi minimiser $\sum_t \left ( P_t - A \right )^2$, et aussi introduire le 
coût de chaque unité photo-voltaique et éolienne. Enfing bref il y a plein de possibilités, 
mais on se retrouve chaque fois avec un problème d'optimisation avec contraintes linéaires
et objectif (la fonction qu'on minimise) linéaire ou quadratique. Tout ça se résout avec avec cvxopt en python par exemple (https://cvxopt.org/).

\vspace{0.5cm}
\noindent
Qu'en penses-tu ?

\vspace{0.5cm}
\noindent
Bisous
\end{document}
